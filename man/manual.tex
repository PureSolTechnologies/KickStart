\documentclass[a4paper,10pt]{book}
\usepackage{color}
\usepackage{graphicx}
\usepackage{amsmath}
\usepackage{makeidx}
\usepackage{nomencl}
\usepackage[margin=10pt,font=small,labelfont=bf,labelsep=newline,singlelinecheck=false]{caption}
\usepackage[german,polutonikogreek,english]{babel}
\usepackage{german}
\usepackage{ifthen}

%%%%%%%%%%%%%%%%%%%%%%%%%%%%%%%%%%%%%%%%%%%%%%%%%%%%%%%%%%%%%%%%%%%%%%%%%%%%%%
\newcommand{\TITLE}{The ``complete'' KickStart manual}
\newcommand{\SUBTITLE}{Users, Administrators and Developers Guide to 
KickStart}
\newcommand{\AUTHOR}{Rick-Rainer Ludwig}
\newcommand{\DATE}{\today}
\newcommand{\VERSION}{1.0.0}
\newcommand{\COMPANY}{PureSol Technologies}
%%%%%%%%%%%%%%%%%%%%%%%%%%%%%%%%%%%%%%%%%%%%%%%%%%%%%%%%%%%%%%%%%%%%%%%%%%%%%%

\newenvironment{definition}[2]
{
  \index{#1}
  \vspace{1ex}
  Definition \textbf{#1}:
  \begin{flushright}
    \begin{minipage}{0.9\textwidth}
      #2
    \end{minipage}
  \end{flushright}
}
%%%%%%%%%%%%%%%%%%%%%%%%%%%%%%%%%%

\bibliographystyle{unsrt}

%\renewcommand{\nomname}{\chapter{Glossary and Register}}
\makenomenclature

%\renewcommand{\indexname}{\chapter{Index}}
\makeindex

\newenvironment{story}[1]{
  {
    \sffamily\small
    \list{}
         {
           \leftmargin1cm
           \rightmargin1cm
         }
       \item\relax\textit{#1}
         \endlist
  }
}

\newenvironment{person}[1]
{\textsc{#1}\index{#1}}

\newenvironment{term}[1]
{\textit{#1}\index{#1}}

\renewenvironment{quote}[3] {
  \index{#2}
  \begin{flushright}
    \begin{minipage}{0.75\textwidth}
      \sffamily
      \small
      \begin{flushright}
	\textit{``#1''}\\-- \person{#2}
	\ifthenelse{\equal{#3}{}}
	{}
	{#3}
      \end{flushright}
    \end{minipage}
  \end{flushright}
}

%%%%%%%%%%%%%%%%%
%%% Farben... %%%
\definecolor{red}{rgb}{1.0,0.0,0.0}
\definecolor{blue}{rgb}{0.0,0.0,1.0}
\definecolor{green}{rgb}{0.0,1.0,0.0}
%%%%%%%%%%%%%%%%%

\begin{document}

\pagestyle{headings}

\newcommand{\bs}{\textbackslash}

\begin{titlepage}
	{
	\centering
	\textbf{\Huge \TITLE}\\[1cm]
	\textbf{\Large \SUBTITLE}\\[2cm]
	\textbf{\AUTHOR}\\
	\textbf{(c) by \COMPANY}\\[1cm]
	\textbf{version: \VERSION}\\
	\textbf{date: \DATE}\\[1cm]
	}
        This is the ``official'' manual for KickStart.
\end{titlepage}

\tableofcontents
\listoffigures
\listoftables

\part{Overview}
\chapter{About KickStart}
KickStart is a Java application launcher framework for easy installation and usage. The most annoying thing in Java is the deployment procedure where a lot
of different Jar files need to be packed and within the manifests need to be coupled together by there CLASSPATH configuration. A simple add, remove or change of different classes or a refactoring within this Jars is always time consuming.

KickStart tries to avoid this by starting itself on a pure JRE installation and collection afterwards all additionally needed libraries and collecting them for an enhanced CLASSPATH. A coupling of Jars is not needed anymore and Jars can be replaced, splitted and merged as wanted, as soon as there are all dependencies in range of KickStart during startup.

\part{Installation and Configuration}
\chapter{Installation of KickStart}
The Installation can be done anywhere wanted. On Unix like System an installation directory like \texttt{/opt/JavaApps} is suitable. Within this directory a recommendation is to create the following subdirectories:
\begin{itemize}
\item \textbf{bin}: Within this directory all executable scripts are located which actually start KickStart and choose the Class in which the needed main() method can be found for startup.
\item \textbf{config}: This directory should be treated as class library directory and is used for configuration files which are placed into the CLASSPATH and loaded as resource.
\item \textbf{extlib}: To \texttt{extlib} all libraries from external vendors should go in.
\item \textbf{lib}: Should be the directory where all classes go in which are from own developmens and which are changed during development cycles.
\item \textbf{log}: Within this directory all log files which are generated by the tools should be located. KickStart itself does not generate any log files.
\end{itemize}
All directory naming is meant to be inside the installation directory of the KickStart framework.

\chapter{System Configuration}

\section{PATH variable}
On Unix one adds KickStart to a global \texttt{PATH} variable by changing \texttt{/etc/profile}, \texttt{/etc/profile}, \texttt{/etc/bash.bashrc} or \texttt{/etc/bash.bashrc.local}. The location depends on the System used. For user related changes the files \texttt{~/.profile} or \texttt{~/.bashrc} should be checked.

For Windows in the System Control Panel the path can be adjusted.

\section{Cron}
To add the KickStart \texttt{bin} directory into the Cron PATH variable modify \texttt{/etc/crontab} and add a line like \texttt{PATH=/usr/bin:/usr/sbin:/sbin:/bin:/usr/lib/news/bin}.

\part{Usage}

\part{Appendices}
\index{Appendices}

% GLOSSARY
\nomenclature{\textbf{ASCII}\index{ASCII}}{American Standard Code for
Information Interchange - Code table for a number to character mapping}
\nomenclature{\textbf{Design Patterns}\index{Design Patterns}}{Design Patterns
(in german: Entwurfsmuster) are generalized approaches to repeating
programming issues. For the most issues there is a fitting approach to solve
the issue with a generalized design. Design patterns are well known with
generalized names and software which implements and documents these patterns
are very easy mainanable. Other programmers have a good chance to understand
the code and the design of the software in short time. (see:
\cite{book:head_first_design_patterns,book:design_patterns})}
\nomenclature{\textbf{Refactoring}\textbf{Refactoring}}{Refactoring is the
process of permanent code improvement even after implementing functionality.
This process is essential to keep the software code in a clean state for later
use. Without Refactoring the code will get out of shape and the implementation
process for new functionallity will become more expensive. Therefore,
Refactoring is a way to keep code maintanable. (see:
\cite{book:the_pragmatic_programmer,book:refactoring})}

\printnomenclature % <-- new version since 4.1!

\bibliography{refs/books}

\printindex

\end{document}
